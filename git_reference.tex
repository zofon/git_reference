\documentclass[a4paper,12pt]{ctexbook}
\usepackage[margin=2cm]{geometry}
\usepackage{graphicx}
\usepackage{subfigure}
\usepackage{float}
\usepackage[colorlinks,linkcolor=black]{hyperref}%colorlinks启用链接颜色,linkcolor指定对应的颜色
\CTEXsetup[format={\Large\bfseries}]{section}  
\usepackage{listings}
%\usepackage[cache=false]{minted}  % 代码高亮X
\usepackage{fontspec}
\usepackage{tikz,xcolor,mwe}
\definecolor{cvgreen}{HTML}{92D14F}
\definecolor{cvgray}{HTML}{D8E4BE}
\definecolor{cvtext}{HTML}{92909B}
\usetikzlibrary{shadows}

\pagestyle{empty}
\begin{document}
\begin{flushleft}
		
\begin{tikzpicture}[remember picture,overlay] 
\fill[cvgreen] (current page.north west) rectangle ([xshift=6cm]current page.south west);% green bar
\fill[cvgray] ([yshift=-11cm]current page.north west) rectangle ([yshift=-17cm]current page.north east); % gray bar

\node[cvtext,right] at ([xshift=6cm,yshift=-13cm]current page.north west) {\Huge Git Reference};
\node[cvtext,above left] at ([xshift=-1cm,yshift=-16.5cm]current page.north east) {\Large\bfseries \today};
% cover photo
\node[inner sep=0pt,below right] (image) at ([xshift=17cm,yshift=-1cm]current page.north west) {\includegraphics[width=3cm]{hainulogo.png}};
% name and address
\node[fill=white,align=center,text width=6.4cm,inner sep=0.8cm,below] (name) at (image.south) {};
\node[text width=15cm,inner sep=0.3cm,below right] at (name.south west){\Large Author:Flynn\\E-Mail:zofon@qq.com};
% attachments
\node[white,text width=5cm,inner sep=0.6cm,above right] at ([yshift=1cm]current page.south west)
{\large\obeylines\textbf{This is for \\The Hi-Net Group}};  
\end{tikzpicture}
\tableofcontents
\chapter{git介绍}
Git是一款免费、开源的分布式版本控制系统,用于敏捷高效地处理任何或小或大的项目。

Git 是 Linus Torvalds 为了帮助管理 Linux 内核开发而开发的一个开放源码的版本控制软件。Torvalds 开始着手开发 Git 是为了作为一种过渡方案来替代 BitKeeper,后者之前一直是 Linux 内核开发人员在全球使用的主要源代码工具。开放源码社区中的有些人觉得BitKeeper 的许可证并不适合开放源码社区的工作,因此 Torvalds 决定着手研究许可证更为灵活的版本控制系统。尽管最初 Git 的开发是为了辅助 Linux 内核开发的过程,但是我们已经发现在很多其他自由软件项目中也使用了 Git。例如 很多 Freedesktop 的项目迁移到了 Git 上。

在国内这样的托管平台有码云,阿里云等等。

\textcolor[rgb]{0.00,1.00,0.25}{本文以BitBucket为例介绍git的使用方法}

\section{BitBucket简介}
BitBucket是一家源代码托管网站,默认的免费账号,可以总共有5个帐户对你的私有库进行读写。他们给非营利组织(NPO)和大学生免费申请无限账号的机会。何为无限账号:Bitbucket提供每个用户无限公开和私有库,唯一限制的是对私有库有读写权限的帐户总数。适合小型的团体协作。\\
官网地址:\url{https://bitbucket.org/}

在我们的使用中发现,BitBucket的传输速度是比较慢的,所以除非和Thomas他们合作写论文,尽量不要使用。
\section{GitHub简介}
GitHub 是一个面向开源及私有软件项目的托管平台,因为只支持 Git 作为唯一的版本库格式进行托管,故名 GitHub。GitHub 于 2008 年 4 月 10 日正式上线,除了 Git 代码仓库托管及基本的 Web 管理界面以外,还提供了订阅、讨论组、文本渲染、在线文件编辑器、协作图谱(报表)、代码片段分享(Gist)等功能。目前,其注册用户已经超过350万,托管版本数量也是非常之多,其中不乏知名开源项目 Ruby on Rails、jQuery、python 等。
官网地址:\url{https://github.com}

GitHub毕竟老牌托管平台,不知道是不是在中国有服务器,但是就我们的使用而言,其传输速度还是比较快的。
\section{阿里云code简介}
官网地址:\url{https://code.aliyun.com/}\\
阿里很有钱,现在正在拓展云服务,所以开几个git服务器是小菜一碟。阿里云的带宽和速度也是杠杠的没话说。
阿里云可以直接使用淘宝账号登录,根据官网提示注册登录即可。
\section{码云简介}
官网地址:\url{https://git.oschina.net/}\\
\hyperref{https://git.oschina.net/}{category}{码云}{码云}(Git@OSC)是开源中国社区团队推出的基于Git的快速的、免费的、稳定的在线代码托管平台,不限制私有库和公有库数量。
因为服务器在中国,所以传输很快,和阿里云相比而言还是要逊色一点,毕竟马老爸。
\section{小结}
上述的这些托管平台中,恐怕要针对不同的情况使用不同的平台了,和Thomas合作写论文,英文论文不能公开,所以只能使用BitBucket,如果要是自己想写文档,那么我比较推荐阿里云code和码云,毕竟是国内的服务器,当然了,如果有开源和分享的精神,部署在GitHub上面也是一个很不错的选择,毕竟传输速度也不是很慢。

\chapter{创建仓库}
\begin{itemize}
  \item 在\url{https://bitbucket.org}上面创建一个账号。
  \item 进入账号之后,点击创建仓库:
        \begin{figure}[H]
        \centering
        \includegraphics[width=15cm]{figures/create_reponsity.jpg}
        \end{figure}

  \item 创建仓库之后,选择仓库类型:
        \begin{figure}[H]
        \centering
        \includegraphics[width=10cm]{figures/create_reponsity_2.jpg}
        \end{figure}

  \item 之后可以参考命令行的相关提示:
        \begin{figure}[H]
        \centering
        \includegraphics[width=15cm]{figures/create_reponsity_3.jpg}
        \end{figure}
\end{itemize}

\section{上传私钥}
上传公钥之后就可以用公钥登陆bitbucket,不需要输入密码。点击右上角的个人图标,进入个人设置:
\begin{figure}[H]
  \centering
  \includegraphics{figures/ssh_set_1.jpg}\\
\end{figure}
然后常规$\rightarrow$安全$\rightarrow$SSH密钥$\rightarrow$添加密钥,按照说明基本没问题。\\
\begin{verbatim}
如下的命令可以将密钥复制到剪切板,比较方便。
安装xclip:sudo apt install xclip
复制到剪切板:cat ~/.ssh/id_rarsa.pub | xclip -sel clip
\end{verbatim}

\section{注意点}
在使用阿里云code的时候,好像必须要上传一个ssh key,不然是没有访问权限的。
(根据我的经验,在git pull的时候出现权限问题,只要上传你的ssh key 就可以解决问题。)


\chapter{团队协作}
Git对于团队协作来说非常重要,尽管有一些学习曲线,还是希望能完全掌握这个技能。
我们可以邀请别人进入自己的仓库,也可以接受邀请进入别人的仓库。
\section{接受别人的邀请}
别人邀请你加入他的仓库,你会收到一封邮件,按照提示进行操作,完成之后进入自己的Bitbucket账号,点击如下按钮:
\begin{figure}[H]
  \centering
  \includegraphics[width=10cm]{figures/Bitbucket_button.jpg}
\end{figure}
然后你就会看到当前你的账号下所有的仓库,点击相关的仓库进入。在右上角可以看到克隆地址。

有两种方法可以克隆仓库到本地,HTTP和SSH。

第一种:HTTP是一种比较基础的方法,不需要上传私钥,找到对应的克隆地址,但是每次push和pull和时候要输入账号和密码,如下:
\begin{figure}[H]
  \centering
  \includegraphics[width=10cm]{figures/HTTP_clone.jpg}
\end{figure}
在本地终端输入如下命令:
\begin{verbatim}
git clone https://zofon@bitbucket.org/zofon/bitbucket_reference.git
\end{verbatim}

第二种:SSH需要上传私钥,但是比较方便,找到对应的克隆地址,如下:
\begin{figure}[H]
  \centering
  \includegraphics[width=10cm]{figures/SSH_clone.jpg}
\end{figure}
在本地终端输入如下命令:
\begin{verbatim}
git clone git@bitbucket.org:zofon/bitbucket_reference.git
\end{verbatim}


\section{pull操作基本步骤}
\textbf{要养成进入仓库先 git pull一下的习惯。}这样子可以省去很多麻烦。在一个人使用仓库的时候,只有一个分支,只需要简单的 git pull 命令即可。当多个人在仓库工作的时候,可以用 git pull --all 命令将全部的分支都pull下来,然后处理。

\section{push操作基本步骤}
在初始化仓库的时候是需要一个
\begin{itemize}
   \item git add test.tex
   \item git commit -m "Describation"
   \item git push
\end{itemize}
如果权限不是最高权限,不能push到master分支上的时候,可以自己在本地新建一个分支,然后将该分支push到仓库。
\begin{itemize}
  \item git branch zhoufeng
  \item git checkout zhoufeng
  \item git add --all
  \item git commit -m "Descriptions"
  \item git push origin zhoufeng
\end{itemize}

\section{邀请别人对你的仓库进行读写}
在不同的平台上面方法有一点区别,在GitHub上邀请的时候需要被邀请者登录确认才算是添加进了仓库。
邀请一个人对你的仓库访问的时候有三种权限:
\begin{description}
	\item[master] 和仓库所有者的权限非常接近,可以push到origin分支。
	\item[develper] 权限次于master,可以向仓库提交内容,但是不能push到origin分支。
	\item[reader] 只能读取仓库的权限。
\end{description}
当然了,在不同的平台中,可能有其他的权限表示方式,反正大概就是这三个等级。

\section{小结}
邀请别人参加到你的项目中时候注意设置被邀请者的权限。


\chapter{其他操作命令}
\section{多分支操作}
\begin{description}
  \item[git branch -a] 查看所有的分支。
  \item[git checkout chengxi] 切换到chengxi的分支上。
  \item[git checkout master] 切换回主分支。
  \item[git merge chengxi] 将chengxi分支合并到当前的分支
  \item[git branch -d chengxi] 这个分支合并完了之后可以删掉了。
\end{description}

\section{解决冲突}
解决冲突有两种方法:
一、merge以前的一个版本,即将之前的版本挪到现在的版本。如果你修改了其中的一个文件,就有可能没有办法使用git merge命令,
此时可以删除本地有冲突的文件。然后使用git pull命令就OK了。

二、使用git rebase丢弃特定版本之后的数据。相对而言第一种方法处理较好。

\section{回退到某个版本}
如果不小心操作出错了,回退到上一个版本或者上几个版本,用如下命令:
\begin{verbatim}
git reset是指将当前head的内容重置,不会留log信息。

git reset HEAD filename
#从暂存区中移除文件

git reset --hard HEAD~3
#会将最新的3次提交全部重置,就像没有提交过一样。

git reset --hard commit (38679ed709fd0a3767b79b93d0fba5bb8dd235f8)
#回退到 38679ed709fd0a3767b79b93d0fba5bb8dd235f8 版本

根据--soft --mixed --hard,会对working tree和index和HEAD进行重置:

git reset --mixed
#此为默认方式,不带任何参数的git reset,即时这种方式,它回退到某个版本,只保留源码,回退commit和index信息

git reset --soft
#回退到某个版本,只回退了commit的信息,不会恢复到index file一级。如果还要提交,直接commit即可

git reset --hard
#彻底回退到某个版本,本地的源码也会变为上一个版本的内容
例如:我要彻底返回在上一次提交以前的版本。git reset --hrad HEAD~1

git reset --hard
#回到上一次提交的版本。
\end{verbatim}
回退到之前的版本之后,可能会出现无法push的情况,这个时候如果是自己一个在使用的话,可以强制push,命令如下:git push -f

\section{常用命令介绍}
\begin{description}
  \item[git clone] 在如下的位置找到克隆的地址,可以使用该命令将云端的仓库克隆到本地。
        \begin{figure}[H]
        \centering
        \includegraphics[width=10cm]{figures/clone_address.jpg}
        \end{figure}
        命令如下:git clone https://zofon@bitbucket.org/dreibh/hu-experiment-newscenario.git
  \item[git remote] 查看远端的信息,只能看到:origin
  \item[git remote -v] 查看远端的信息,可以看到:\verb|origin  ssh://git@bitbucket.org/zofon/buffer_model.git (fetch)|
  \item[git reflog] 查看提交的版本信息,比较关键的命令。
  \item[git status] 查看当前的提交状态,一般提交之后可以使用这个命令查看。

  \item[git add a.tex] 将改动的文件添加到提交,命令如下:git add *tex ,即自己确定要上传某个文件。
  \item[git add .] 添加当前文件夹的内容到提交,相对--all而言,这个不是递归的。
  \item[git add --all] 递归提交,将文件的所有变动都添加到提交。用于改动文件较多的情况。

  \item[git rm] 删除云端的指定文件,在这之后也是需要commit一下的。

  \item[git commit -am "描述"] 提交命令,-m的意思就是添加描述,-a的意思就是全部提交,比如说之前有添加了没有提交的。
  \item[git commit -m  "描述"] 提交命令,-m的意思就是添加描述,提交这次添加的内容。

  \item[git push -u origin master] 这个在第一次提交的时候使用,第一次需要指定提交的分支。
  \item[git push] 将提交的内容push到云端。
\end{description}

\chapter{常见问题}
\section{git commit -m 'Initial commit' 时候出现错误}
错误信息如下:
\begin{verbatim}
*** Please tell me who you are.

Run

  git config --global user.email "you@example.com"
  git config --global user.name "Your Name"

to set your account's default identity.
Omit --global to set the identity only in this repository.

fatal: unable to auto-detect email address (got 'Administrator@Sc-201608242018.(none)')
\end{verbatim}
这个问题是因为你没有设置相关的用户账号的名字。设置一下即可:
\begin{verbatim}
git config --global user.email "969493314@qq.com"
git config --global user.name "zofon"
\end{verbatim}

\section{git push -u origin master 时候出现错误}
错误信息如下:
\begin{verbatim}
The authenticity of host 'bitbucket.org (104.192.143.1)' can't be established.
RSA key fingerprint is SHA256:zzXQOXSRBEiUtuE8AikJYKwbHaxvSc0ojez9YXaGp1A.
Are you sure you want to continue connecting (yes/no)? yes
Warning: Permanently added 'bitbucket.org,104.192.143.1' (RSA) to the list of known hosts.
Permission denied (publickey).
fatal: Could not read from remote repository.

Please make sure you have the correct access rights
and the repository exists.
\end{verbatim}
这个问题是因为没有添加公钥到Bitbucket上面。添加一个密钥即可。其实这里我也不知道为什么。。。只是这样子可以解决这个问题。

根据我的经验,在GitHub、码云上使用https克隆时不会出现这个问题,阿里云会出现这个问题。
\section{git clone 某个仓库时候出现错误}
错误信息如下:
\begin{verbatim}
remote: Counting objects: 4592517, done.
remote: Compressing objects: 100% (1140430/1140430), done.
error: RPC failed; result=56, HTTP code = 2008.82 MiB | 4.72 MiB/s
fatal: The remote end hung up unexpectedly
fatal: early EOF
fatal: index-pack failed
\end{verbatim}
这个问题是因为克隆的仓库的大小超过了Git的传输字节限制,修改其值即可(这里的单位是Byte):\\
\begin{verbatim}
git config --global http.postBuffer  524288000
\end{verbatim}
524288000就是512MiB\\
所以说要注意自己的仓库的大小。事实上,仓库是不需要多大的。

\section{git pull的时候产生错误}
错误信息如下:
\begin{verbatim}
$ git pull
remote: Counting objects: 27, done.
remote: Compressing objects: 100% (27/27), done.
Connection to bitbucket.org closed by remote host.
fatal: The remote end hung up unexpectedly
fatal: early EOF
fatal: unpack-objects failed
\end{verbatim}
这个问题我也不清楚是什么情况,解决方法就是彻底删掉这个版本库,然后重新克隆。有效。
\end{flushleft}
\end{document}
